\PassOptionsToPackage{unicode=true}{hyperref} % options for packages loaded elsewhere
\PassOptionsToPackage{hyphens}{url}
%
\documentclass[]{article}
\usepackage{lmodern}
\usepackage{amssymb,amsmath}
\usepackage{ifxetex,ifluatex}
\usepackage{fixltx2e} % provides \textsubscript
\ifnum 0\ifxetex 1\fi\ifluatex 1\fi=0 % if pdftex
  \usepackage[T1]{fontenc}
  \usepackage[utf8]{inputenc}
  \usepackage{textcomp} % provides euro and other symbols
\else % if luatex or xelatex
  \usepackage{unicode-math}
  \defaultfontfeatures{Ligatures=TeX,Scale=MatchLowercase}
\fi
% use upquote if available, for straight quotes in verbatim environments
\IfFileExists{upquote.sty}{\usepackage{upquote}}{}
% use microtype if available
\IfFileExists{microtype.sty}{%
\usepackage[]{microtype}
\UseMicrotypeSet[protrusion]{basicmath} % disable protrusion for tt fonts
}{}
\IfFileExists{parskip.sty}{%
\usepackage{parskip}
}{% else
\setlength{\parindent}{0pt}
\setlength{\parskip}{6pt plus 2pt minus 1pt}
}
\usepackage{hyperref}
\hypersetup{
            pdftitle={vastws01},
            pdfauthor={Yuki Kanamori},
            pdfborder={0 0 0},
            breaklinks=true}
\urlstyle{same}  % don't use monospace font for urls
\usepackage[margin=1in]{geometry}
\usepackage{graphicx,grffile}
\makeatletter
\def\maxwidth{\ifdim\Gin@nat@width>\linewidth\linewidth\else\Gin@nat@width\fi}
\def\maxheight{\ifdim\Gin@nat@height>\textheight\textheight\else\Gin@nat@height\fi}
\makeatother
% Scale images if necessary, so that they will not overflow the page
% margins by default, and it is still possible to overwrite the defaults
% using explicit options in \includegraphics[width, height, ...]{}
\setkeys{Gin}{width=\maxwidth,height=\maxheight,keepaspectratio}
\setlength{\emergencystretch}{3em}  % prevent overfull lines
\providecommand{\tightlist}{%
  \setlength{\itemsep}{0pt}\setlength{\parskip}{0pt}}
\setcounter{secnumdepth}{0}
% Redefines (sub)paragraphs to behave more like sections
\ifx\paragraph\undefined\else
\let\oldparagraph\paragraph
\renewcommand{\paragraph}[1]{\oldparagraph{#1}\mbox{}}
\fi
\ifx\subparagraph\undefined\else
\let\oldsubparagraph\subparagraph
\renewcommand{\subparagraph}[1]{\oldsubparagraph{#1}\mbox{}}
\fi

% set default figure placement to htbp
\makeatletter
\def\fps@figure{htbp}
\makeatother


\title{vastws01}
\author{Yuki Kanamori}
\date{2020/1/17}

\begin{document}
\maketitle

\hypertarget{vast-workshop-2020}{%
\section{VAST workshop 2020}\label{vast-workshop-2020}}

執筆: 金森由妃(研究支援@中央水研)\\
\href{mailto:kana.yuki@fra.affrac.go.jp}{\nolinkurl{kana.yuki@fra.affrac.go.jp}}

\hypertarget{part-1-ux5e74ux52b9ux679cux3060ux3051ux306eux30e2ux30c7ux30eb}{%
\subsection{Part 1:
年効果だけのモデル}\label{part-1-ux5e74ux52b9ux679cux3060ux3051ux306eux30e2ux30c7ux30eb}}

まずは年効果のみを入れたモデルを単一種のデータに適用してみる.モデルは,・・・.\\
本解析で必要な情報は『年,CPUE(or
アバンダンスと努力量),緯度,経度』である. \#\#\# 0.
フォルダとデータの作成 1.
ワークショップ用のフォルダ『vastws』を任意の場所に作成し,パスを確認する.\\
VASTのアウトプットは容量が大きいため,\textbf{フォルダをデスクトップに作成することはお勧めしない.}\\
2. 作成したフォルダに,解析で用いるデータ(csvファイルなど)を入れる 3.
確認したパスを以下のように入力し,作成したフォルダを作業ディレクトリとして設定する

\begin{verbatim}
dirname = "/Users/Yuki/Dropbox/vastws"    
setwd("dir = dirname")
\end{verbatim}

\begin{enumerate}
\def\labelenumi{\arabic{enumi}.}
\setcounter{enumi}{3}
\tightlist
\item
  解析で用いるパッケージを呼び出す
\end{enumerate}

\begin{verbatim}
require(VAST)
require(TMB)
\end{verbatim}

\begin{enumerate}
\def\labelenumi{\arabic{enumi}.}
\setcounter{enumi}{4}
\tightlist
\item
  データを読み込み,オブジェクト名をdfとする.例えばcsvファイルでは,
  \texttt{df\ =\ read.csv("\#\#\#\#.csv")}
\item
  各列に『年,CPUE(あるいは,アバンダンスと努力量),緯度,経度』が入ったデータフレーム(tidyデータ)を作成する.オブジェクト名は,dfとしたままでよい.
\end{enumerate}

\hypertarget{ux5404ux7a2eux8a2dux5b9a}{%
\subsubsection{1. 各種設定}\label{ux5404ux7a2eux8a2dux5b9a}}

\hypertarget{cppux30d5ux30a1ux30a4ux30ebux306eux30d0ux30fcux30b8ux30e7ux30f3ux3092ux6307ux5b9a}{%
\paragraph{1.1
cppファイルのバージョンを指定}\label{cppux30d5ux30a1ux30a4ux30ebux306eux30d0ux30fcux30b8ux30e7ux30f3ux3092ux6307ux5b9a}}

\begin{itemize}
\tightlist
\item
  cppファイルとはTMBを動かすコードのことで,C++言語で書かれている
\item
  cppファイルのバージョンは,\textbf{VASTやTMBなどのバージョンとは別物}
\item
  最新版では色々なオプションがあるが,CPUE標準化では使わない場合が多い
\end{itemize}

\begin{verbatim}
# 最新版のcppファイルを指定する時
Version = get_latest_version(package = "VAST")
\end{verbatim}

\begin{itemize}
\tightlist
\item
  MacとLinuxでは最新版のcppファイルをコンパイルできないバグが発生しているため,テストコードを走らせたバージョンを以下のように指定する
\end{itemize}

\begin{verbatim}
Version = "VAST_v4_2_0"
\end{verbatim}

\begin{itemize}
\tightlist
\item
  \texttt{VAST\_v4\_2\_0}あたりが安定しているっぽい 
\end{itemize}

\hypertarget{ux7a7aux9593ux306eux8a2dux5b9a}{%
\paragraph{1.2 空間の設定}\label{ux7a7aux9593ux306eux8a2dux5b9a}}

K平均法によるknot決めを行う

\begin{verbatim}
Method = c("Grid", "Mesh", "Spherical_mesh")[#]
\end{verbatim}

\begin{itemize}
\tightlist
\item
  データの観測点が空間的に均一な場合(例えば,格子点上に観測点が存在する)には,
  \texttt{Method\ =\ c("Grid",\ "Mesh",\ "Spherical\_mesh"){[}1{]}}を選択する
\item
  データの観測点が空間的に不均一な場合には,
\end{itemize}

\begin{verbatim}
# 観測点が狭い範囲にある場合(例えば,日本近海)
Method = c("Grid", "Mesh", "Spherical_mesh")[2]
# 観測点が全球に渡る場合
Method = c("Grid", "Mesh", "Spherical_mesh")[3]
\end{verbatim}

とする.Thorson
(2019)は,Meshを使った場合でも感度分析的にgridでも解析することを勧めている.
* ここでの設定について理解するためには,Gaussian field,Gaussian Markov
Random
Field,Matérn関数,INLA,SPDE,有限要素法などを勉強する必要がある(とっても難しい).

\begin{verbatim}
# 変更の必要なし
Kmeans_Config = list("randomseed" = 1, "nstart" = 100, "iter.max" = 1000)
\end{verbatim}

\begin{verbatim}
grid_size_km = 2.5
\end{verbatim}

\begin{itemize}
\tightlist
\item
  MethodがGridの場合に必要な情報
\item
  Meshの場合には関係ないが,NULLとすると『2.3 derived objects for
  spatio-temporal estimation』でエラーが出るため\textbf{触らない}
\end{itemize}

\begin{verbatim}
# knotの数の指定
n_x = 100
\end{verbatim}

\begin{itemize}
\tightlist
\item
  Thorson (2019)は100以上を推奨
\item
  knot数が多いほど滑らかに近似されるためAICは下がるが,計算負荷が大きくなる
\end{itemize}

\hypertarget{ux30e2ux30c7ux30ebux306eux8a2dux5b9a}{%
\paragraph{1.3
モデルの設定}\label{ux30e2ux30c7ux30ebux306eux8a2dux5b9a}}

\begin{itemize}
\tightlist
\item
  プログラムコードの中でもっとも重要な部分で,解析するモデルについて『因子分析の因子数・時間の扱い・分散・観測誤差とリンク関数』を設定する.\\
\item
  ギリシャ文字とギリシャ文字の直後の数字はVASTのモデル式と対応している.例えば,Beta1は遭遇確率の年効果,Beta2は遭遇確率が
  \textgreater{}0である場合の密度の年効果を表す.
\end{itemize}

\begin{verbatim}
FieldConfig = c(Omega1 = 1, Epsilon1 = 1, Omega2 = 1, Epsilon2 = 1)
\end{verbatim}

\begin{itemize}
\tightlist
\item
  因子分析の因子数.上限はカテゴリー数(種,年齢,体長などの数).今回は単一種を解析するため,最大数は1.
\end{itemize}

\begin{verbatim}
RhoConfig = c(Beta1 = 0, Beta2 = 0, Epsilon1 = 0, Epsilon2 = 0)
\end{verbatim}

\begin{itemize}
\tightlist
\item
  今回は年を固定効果,時空間のランダム効果の年は独立と考えている
\item
  \(\beta\)には,分散が年で変わる(= 1),ランダムウォーク(= 2),定数(=
  3),AR(= 4)が選択できる.
\item
  \(\epsilon\)には,ランダムウォーク(= 2),AR(= 4)が選択できる.
\end{itemize}

\begin{verbatim}
OverdispersionConfig = c("Eta1" = 0, "Eta2" = 0)
\end{verbatim}

\begin{itemize}
\tightlist
\item
  詳細はPart 2で紹介するため,とりあえず0にする
\end{itemize}

\begin{verbatim}
ObsModel = c(PostDist = , Link = )
\end{verbatim}

\begin{itemize}
\tightlist
\item
  観察誤差の分布とリンク関数についての設定.非常にたくさんの選択肢がある.詳細は?make\_dataを参照されたい
\item
  ここでは簡単のため,代表的な場合を紹介する \#
  データの種類とパラメータの選択を表にまとめる 
\end{itemize}

\hypertarget{ux30c7ux30fcux30bfux306eux7bc4ux56f21}{%
\paragraph{1.4
データの範囲1}\label{ux30c7ux30fcux30bfux306eux7bc4ux56f21}}

\begin{verbatim}
strata.limits = data.frame('STRATA'="All_areas")
\end{verbatim}

\begin{itemize}
\tightlist
\item
  変更の必要はない 
\end{itemize}

\hypertarget{ux30c7ux30fcux30bfux306eux7bc4ux56f22}{%
\paragraph{1.5
データの範囲2}\label{ux30c7ux30fcux30bfux306eux7bc4ux56f22}}

\begin{verbatim}
Region = "other"
\end{verbatim}

\begin{itemize}
\tightlist
\item
  自分のデータを解析する場合は,``other''に変更
\item
  FishStatsUtilsに入っているテストデータを解析する時のみ,適切な地域を選択する.
\end{itemize}

\hypertarget{ux8a2dux5b9aux306eux4fddux5b58}{%
\paragraph{1.6 設定の保存}\label{ux8a2dux5b9aux306eux4fddux5b58}}

\begin{verbatim}
DateFile = paste0(getwd(),'/VAST_output/')
dir.create(DateFile)
Record = list(Version = Version,
              Method = Method,
              grid_size_km = grid_size_km,
              n_x = n_x,
              FieldConfig = FieldConfig,
              RhoConfig = RhoConfig,
              OverdispersionConfig = OverdispersionConfig,
              ObsModel = ObsModel,
              Kmeans_Config = Kmeans_Config,
              Region = Region,
              strata.limits = strata.limits)
setwd(dir = DateFile)
save(Record, file = file.path(DateFile, "Record.RData"))
capture.output(Record, file = paste0(DateFile, "/Record.txt"))
\end{verbatim}

\begin{itemize}
\tightlist
\item
  作業ディレクトリの直下に,\texttt{VAST\_output}というフォルダが作成され,結果が入れられていく.
\item
  デフォルトのままだとフォルダ名が解析ごとに同じになるため,\textbf{解析結果が上書き保存されてしまう}
\item
  \texttt{paste0(getwd(),\ "/vast",\ Sys.Date(),\ "\_lnorm\_log",\ n\_x,\ sakana)}などどしておくと,フォルダ名を見ただけで『いつ,どんなモデルで,どれくらいのknot数で,どの魚種を解析した結果なのか』が分かる
\end{itemize}

\hypertarget{vastux306bux5408ux308fux305bux305fux30c7ux30fcux30bfux30bbux30c3ux30c8ux306eux6e96ux5099}{%
\subsubsection{2.
VASTに合わせたデータセットの準備}\label{vastux306bux5408ux308fux305bux305fux30c7ux30fcux30bfux30bbux30c3ux30c8ux306eux6e96ux5099}}

\hypertarget{ux30c7ux30fcux30bfux30d5ux30ecux30fcux30e0ux306eux4f5cux6210}{%
\paragraph{2.1
データフレームの作成}\label{ux30c7ux30fcux30bfux30d5ux30ecux30fcux30e0ux306eux4f5cux6210}}

\begin{verbatim}
head(df)

# CPUEデータの時
Data_Geostat = df %>%
  mutate(Year = year,
         Lon = lon,
         Lat = lat,
         Catch_KG = cpue)
# アバンダンスと努力量データの時
Data_Geostat = df %>%
  mutate(Year = year,
         Lon = lon,
         Lat = lat,
         Catch_KG = abundance,
         AreaSwept_km2 = effort)
\end{verbatim}

\begin{itemize}
\tightlist
\item
  \textbf{VASTに渡すデータのオブジェクト名は,必ずData\_Geostat}
\item
  \textbf{列名はオリジナルで作成せず,VAStのデフォルトに合わせる.また,列名はキャメルケース(大文字始まり)で書く}
\item
  Data\_Geostatでない場合,列名をオリジナルで作成した場合,列名がキャメルケースでない場合は,以降のコードを修正する必要が出てくる(関数の中身も修正しなければいけないので,めちゃくちゃ大変)
\end{itemize}

\hypertarget{ux30c7ux30fcux30bfux30d5ux30ecux30fcux30e0ux304bux3089ux4f4dux7f6eux60c5ux5831ux3092ux53d6ux5f97}{%
\paragraph{2.2
データフレームから位置情報を取得}\label{ux30c7ux30fcux30bfux30d5ux30ecux30fcux30e0ux304bux3089ux4f4dux7f6eux60c5ux5831ux3092ux53d6ux5f97}}

\begin{verbatim}
Extrapolation_List = FishStatsUtils::make_extrapolation_info(
  Regio = Region, #zone range in Japan is 51:56
  strata.limits = strata.limits,
  observations_LL = Data_Geostat[, c("Lat", "Lon")],
)
\end{verbatim}

\begin{itemize}
\tightlist
\item
  緯度経度をUTM(Universal Transverse Mercator)座標へ変換している
\item
  データフレームから検出した位置情報(zone)を教えてくれるので確認する
\end{itemize}

\begin{verbatim}
Using strata 1
convUL: For the UTM conversion, automatically detected zone 9.   
convUL: Converting coordinates within the northern hemisphere.
# 日本は,51~56の範囲に入る
# この表示はエラーではない
\end{verbatim}

\hypertarget{ux89b3ux6e2cux70b9ux3092knotux306bux5909ux63db}{%
\paragraph{2.4
観測点をknotに変換}\label{ux89b3ux6e2cux70b9ux3092knotux306bux5909ux63db}}

\begin{verbatim}
Spatial_List = FishStatsUtils::make_spatial_info(
  n_x = n_x,
  Lon = Data_Geostat[, "Lon"],
  Lat = Data_Geostat[, "Lat"],
  Extrapolation_List = Extrapolation_List,
  Method = Method,
  grid_size_km = grid_size_km,
  randomseed = Kmeans_Config[["randomseed"]],
  nstart = Kmeans_Config[["nstart"]],
  iter.max = Kmeans_Config[["iter.max"]],
  #fine_scale = TRUE,
  DirPath = DateFile,
  Save_Results = TRUE)
\end{verbatim}

\begin{itemize}
\tightlist
\item
  『1.2 空間の設定』の情報を使っている
\end{itemize}

\begin{verbatim}
convUL: Converting coordinates within the northern hemisphere.  
convUL: For the UTM conversion, used zone 9 as specified  
convUL: Converting coordinates within the northern hemisphere.  
convUL: For the UTM conversion, used zone 9 as specified  
Num=1 Current_Best=Inf New=172166.9  
・
・
・
convUL: Converting coordinates within the northern hemisphere.  
convUL: Converting coordinates within the northern hemisphere.  
# これもエラーではない
\end{verbatim}

\hypertarget{ux30c7ux30fcux30bfux30d5ux30ecux30fcux30e0ux306eux4fddux5b58}{%
\paragraph{2.5
データフレームの保存}\label{ux30c7ux30fcux30bfux30d5ux30ecux30fcux30e0ux306eux4fddux5b58}}

ggvastで描画するためのオリジナルコード

\begin{verbatim}
Data_Geostat = cbind(Data_Geostat,
                     knot_i = Spatial_List[["knot_i"]],
                     zone = Extrapolation_List[["zone"]] #
                     )    
write.csv(Data_Geostat, "Data_Geostat.csv")
\end{verbatim}

\hypertarget{ux30d1ux30e9ux30e1ux30fcux30bfux306eux8a2dux5b9a}{%
\subsubsection{3.
パラメータの設定}\label{ux30d1ux30e9ux30e1ux30fcux30bfux306eux8a2dux5b9a}}

\hypertarget{tmbux306bux6e21ux3059ux30c7ux30fcux30bfux3092ux4f5cux6210ux3059ux308b}{%
\paragraph{3.1
TMBに渡すデータを作成する}\label{tmbux306bux6e21ux3059ux30c7ux30fcux30bfux3092ux4f5cux6210ux3059ux308b}}

\begin{verbatim}
TmbData = make_data(
  Version = Version,
  FieldConfig = FieldConfig,
  OverdispersionConfig = OverdispersionConfig,
  RhoConfig = RhoConfig,
  ObsModel = ObsModel,
  c_iz = rep(0, nrow(Data_Geostat)),
  b_i = Data_Geostat[, 'Catch_KG'],
  a_i = Data_Geostat[, 'AreaSwept_km2'], # CPUEデータの場合は不要
  s_i = Data_Geostat[, 'knot_i'] - 1,
  t_i = Data_Geostat[, 'Year'],
  spatial_list = Spatial_List,
  Options = Options,
  Aniso = TRUE
)
\end{verbatim}

\hypertarget{ux5f15ux6570ux306eux8868ux3092ux5165ux308cux308b}{%
\section{引数の表を入れる}\label{ux5f15ux6570ux306eux8868ux3092ux5165ux308cux308b}}

\begin{itemize}
\tightlist
\item
  その他については,?make\_dataを参照
\end{itemize}

\begin{verbatim}
FieldConfig_input is:  
Component_1 Component_2  
Omega Epsilon
Beta OverdispersionConfig_input is: Eta1 Eta2
1 1 1 1
-2 -2
Calculating range shift for stratum #1:
\end{verbatim}

\hypertarget{ux306eux6642ux306bux3064ux3044ux3066ux5165ux308cux308b}{%
\section{100\%の時について入れる}\label{ux306eux6642ux306bux3064ux3044ux3066ux5165ux308cux308b}}

\hypertarget{ux30d1ux30e9ux30e1ux30fcux30bfux30eaux30b9ux30c8ux3092ux4f5cux6210}{%
\paragraph{3.2
パラメータリストを作成}\label{ux30d1ux30e9ux30e1ux30fcux30bfux30eaux30b9ux30c8ux3092ux4f5cux6210}}

\begin{verbatim}
TmbList = VAST::make_model(TmbData = TmbData,
                           RunDir = DateFile,
                           Version = Version,
                           RhoConfig = RhoConfig,
                           loc_x = Spatial_List$loc_x,
                           Method = Spatial_List$Method)
\end{verbatim}

\begin{itemize}
\tightlist
\item
  『1.1 cppファイルのバージョン』で指定したcppファイルをコンパイルする.
\item
  推定するパラメータが列挙されるので,合っているか確認
\item
  positive
  catchのモデルでは,\{ギリシャ文字\}2しか推定する必要が無いにも関わらず,\{ギリシャ文字\}1も推定パラメータとして列挙されることがある(make\_model()のバグ?).その場合,解析がうまくいかなくなる可能性があるので,以下のようにして不要なパラメータを除去する
  \# パラメータの抜き方 \# パラメータについて表? 
\end{itemize}

\hypertarget{ux30d1ux30e9ux30e1ux30fcux30bfux306eux63a8ux5b9a}{%
\paragraph{3.3
パラメータの推定}\label{ux30d1ux30e9ux30e1ux30fcux30bfux306eux63a8ux5b9a}}

\begin{verbatim}
# 何も変更しない
Obj = TmbList[["Obj"]]
Opt = TMBhelper::fit_tmb(obj = Obj,
                          lower = TmbList[["Lower"]],
                          upper = TmbList[["Upper"]],
                          getsd = TRUE,
                          savedir = DateFile,
                          bias.correct = TRUE)
\end{verbatim}

\begin{verbatim}
Constructing atomic D_lgamma
Optimizing tape... Done
iter: 1 value: 13012.14 mgc: 36.81998 ustep: 1
iter: 2 value: 12951.89 mgc: 9.56431 ustep: 1
iter: 3 value: 12949.05 mgc: 2.199174 ustep: 1
Matching hessian patterns... Done
outer mgc: 3081.279
・
・
・
iter: 1 mgc: 2.867521e-11
outer mgc: 0.004092186
Optimizing tape... Done
iter: 1 mgc: 2.867521e-11
Matching hessian patterns... Done
outer mgc: 31832.82
#########################
The model is likely not converged
#########################
\end{verbatim}

\begin{itemize}
\tightlist
\item
  『収束していない』と出るが,モデル診断で問題が無い場合でも出てくるメッセージなので,『終わったよ』の合図くらいに思っておけばよい
\end{itemize}

\hypertarget{ux63a8ux5b9aux7d50ux679cux306eux4fddux5b58}{%
\paragraph{3.4
推定結果の保存}\label{ux63a8ux5b9aux7d50ux679cux306eux4fddux5b58}}

\begin{verbatim}
Report = Obj$report()
Save = list("Opt" = Opt,
            "Report" = Report,
            "ParHat" = Obj$env$parList(Opt$par),
            "TmbData" = TmbData)
save(Save, file = paste0(DateFile,"/Save.RData"))
\end{verbatim}

\begin{itemize}
\tightlist
\item
  作業ディレクトリに推定結果が\texttt{Save.RData}として保存される 
\end{itemize}

\hypertarget{ux63cfux753b}{%
\subsubsection{4. 描画}\label{ux63cfux753b}}

何も考えずに全て実行する

\begin{verbatim}
# 4.1 Plot data
plot_data(Extrapolation_List = Extrapolation_List,
          Spatial_List = Spatial_List,
          Data_Geostat = Data_Geostat,
          PlotDir = DateFile)

# 4.2 Convergence
pander::pandoc.table(Opt$diagnostics[, c('Param','Lower','MLE',
                                         'Upper','final_gradient')])

# 4.3 Diagnostics for encounter-probability component
Enc_prob = plot_encounter_diagnostic(Report = Report,
                                     Data_Geostat = Data_Geostat,
                                     DirName = DateFile)

# 4.4 Diagnostics for positive-catch-rate component
Q = plot_quantile_diagnostic(TmbData = TmbData,
                             Report = Report,
                             FileName_PP = "Posterior_Predictive",
                             FileName_Phist = "Posterior_Predictive-Histogram",
                             FileName_QQ = "Q-Q_plot",
                             FileName_Qhist = "Q-Q_hist",
                             DateFile = DateFile )
# 4.5 Diagnostics for plotting residuals on a map
MapDetails_List = make_map_info("Region" = Region,
                                "spatial_list" = Spatial_List,
                                "Extrapolation_List" = Extrapolation_List)
Year_Set = seq(min(Data_Geostat[,'Year']), max(Data_Geostat[,'Year']))
Years2Include = which(Year_Set %in% sort(unique(Data_Geostat[,'Year'])))

# FishStatsUtils(2.3.4)を使っている場合は#の行も入れる
# それ以前のバージョンのFishStatsUtilsを使っている場合は#の行をコメントアウトする
plot_residuals(Lat_i = Data_Geostat[,'Lat'],
               Lon_i = Data_Geostat[,'Lon'],
               TmbData = TmbData,
               Report = Report,
               Q = Q,
               savedir = DateFile,
               spatial_list = Spatial_List, #
               extrapolation_list = Extrapolation_List, #
               MappingDetails = MapDetails_List[["MappingDetails"]],
               PlotDF = MapDetails_List[["PlotDF"]],
               MapSizeRatio = MapDetails_List[["MapSizeRatio"]],
               Xlim = MapDetails_List[["Xlim"]],
               Ylim = MapDetails_List[["Ylim"]],
               FileName = DateFile,
               Year_Set = Year_Set,
               Years2Include = Years2Include,
               Rotate = MapDetails_List[["Rotate"]],
               Cex = MapDetails_List[["Cex"]],
               Legend = MapDetails_List[["Legend"]],
               zone = MapDetails_List[["Zone"]],
               mar = c(0,0,2,0),
               oma = c(3.5,3.5,0,0),
               cex = 1.8)

# 4.6 Direction of "geometric anisotropy"
plot_anisotropy(FileName = paste0(DateFile,"Aniso.png"),
                Report = Report,
                TmbData = TmbData)

# 4.7 Density surface for each year
Dens_xt = plot_maps(plot_set = c(3),
                    MappingDetails = MapDetails_List[["MappingDetails"]],
                    Report = Report,
                    Sdreport = Opt$SD,
                    PlotDF = MapDetails_List[["PlotDF"]],
                    MapSizeRatio = MapDetails_List[["MapSizeRatio"]],
                    Xlim = MapDetails_List[["Xlim"]],
                    Ylim = MapDetails_List[["Ylim"]],
                    FileName = DateFile,
                    Year_Set = Year_Set,
                    Years2Include = Years2Include,
                    Rotate = MapDetails_List[["Rotate"]],
                    Cex = MapDetails_List[["Cex"]],
                    Legend = MapDetails_List[["Legend"]],
                    zone = MapDetails_List[["Zone"]],
                    mar = c(0,0,2,0),
                    oma = c(3.5,3.5,0,0),
                    cex = 1.8,
                    plot_legend_fig = FALSE)
Dens_DF = cbind("Density" = as.vector(Dens_xt),
                "Year" = Year_Set[col(Dens_xt)],
                "E_km" = Spatial_List$MeshList$loc_x[row(Dens_xt),'E_km'],
                "N_km" = Spatial_List$MeshList$loc_x[row(Dens_xt),'N_km'])
pander::pandoc.table(Dens_DF[1:6,], digits=3)

# 4.8 Index of abundance
Index = plot_biomass_index(DirName = DateFile,
                           TmbData = TmbData,
                           Sdreport = Opt[["SD"]],
                           Year_Set = Year_Set,
                           Years2Include = Years2Include,
                           use_biascorr = TRUE)
pander::pandoc.table(Index$Table[,c("Year","Fleet","Estimate_metric_tons",
                                    "SD_log","SD_mt")] )
# 4.9 Center of gravity and range expansion/contraction
plot_range_index(Report = Report,
                 TmbData = TmbData,
                 Sdreport = Opt[["SD"]],
                 Znames = colnames(TmbData$Z_xm),
                 PlotDir = DateFile,
                 Year_Set = Year_Set)
\end{verbatim}

\begin{itemize}
\tightlist
\item
  4.7では推定相対密度のマップが作成される.\texttt{plot\_set\ =\ c()}を変えると,推定相対密度以外のマップも作成可能.詳細は\texttt{?plot\_map}
\item
  バイアスコレクションは必須(Thorson \& ristensen
  2016)なので,4.8では\texttt{use\_biascorr\ =\ TRUE}にする
\item
  4.8と4.9で以下のようなメッセージが出るが,エラーではない
\end{itemize}

\begin{verbatim}
# 4.7
Using bias-corrected estimates for abundance index (natural-scale)...  
Using bias-corrected estimates for abundance index (log-scale)...
\end{verbatim}

\begin{verbatim}
# 4.9
Plotting center-of-gravity...    
Using bias-corrected estimates for center of gravity...  
Plotting effective area occupied...  
Using bias-corrected estimates for effective area occupied (natural scale)...  
Using bias-corrected estimates for effective area occupied (log scale)...
\end{verbatim}

\hypertarget{ux30a2ux30a6ux30c8ux30d7ux30c3ux30c8ux306eux898bux65b9}{%
\subsubsection{5.
アウトプットの見方}\label{ux30a2ux30a6ux30c8ux30d7ux30c3ux30c8ux306eux898bux65b9}}

『4.
描画』で作成されたアウトプットについていくつか紹介する.全てを紹介することはできないので,VASTのgithubの『deprecated\_examples』フォルダに入っている資料(ワークショップHPのマニュアルのリンク先)を参照されたい
\#\#\#\# 5.1 解析したデータの空間情報
\textbf{\texttt{Data\_and\_knots.png}} *
上の図2つが解析した空間範囲のマップ * 下の図がknotの位置

\hypertarget{ux30e2ux30c7ux30ebux8a3aux65ad}{%
\paragraph{5.2 モデル診断}\label{ux30e2ux30c7ux30ebux8a3aux65ad}}

\textbf{\texttt{parameter\_estimates.txt}} *
パラメータの推定値が入っている *
\texttt{\$diagnostics}のMLE列の値がLowerとUpperに近くなっていないか,final\_gradient列の値が0に近くなっているかが収束の判断材料となる

\textbf{\texttt{QQ\_Fnフォルダ}} *
\texttt{Posterior\_Predictive-Histogram-1.jpg}が\(y = x\)に近いかどうかが収束の判断材料となる

\textbf{\texttt{Diag-\/-Encounter\_prob.png}} *
ピンクのリボンは95\%信頼区間

\hypertarget{ux63a8ux5b9aux8cc7ux6e90ux91cfux6307ux6a19ux5024ux306eux5e74ux5909ux5316}{%
\paragraph{5.3
推定資源量指標値の年変化}\label{ux63a8ux5b9aux8cc7ux6e90ux91cfux6307ux6a19ux5024ux306eux5e74ux5909ux5316}}

\textbf{\texttt{Index-Biomass.png}} * 推定資源量指数の平均値とSD *
推定資源量指数とは各knotの推定相対密度に各knotの面積を掛けたもの.詳細はThorson(2019)を参照されたい

\textbf{\texttt{Table\_for\_SS3.csv}} * 『Index-Biomass.png』の元データ

\hypertarget{ux63a8ux5b9aux76f8ux5bfeux5bc6ux5ea6ux306eux30deux30c3ux30d7}{%
\paragraph{5.4
推定相対密度のマップ}\label{ux63a8ux5b9aux76f8ux5bfeux5bc6ux5ea6ux306eux30deux30c3ux30d7}}

\textbf{\texttt{Dens.png}} * 赤いほど相対密度が高いことを表す

\hypertarget{ux91cdux5fc3ux306eux5909ux5316}{%
\paragraph{5.5 重心の変化}\label{ux91cdux5fc3ux306eux5909ux5316}}

\textbf{\texttt{center\_of\_gravity.png}} *
『Dens.png』のデータから重心を計算し,年変化を描画したもの *
重心の算出式はThorson(2019)を参照されたい

\hypertarget{ux6709ux52b9ux9762ux7a4d}{%
\paragraph{5.6 有効面積}\label{ux6709ux52b9ux9762ux7a4d}}

\textbf{\texttt{Effective\_Area.png}} *
算出式はThorson(2019)を参照されたい

\hypertarget{anisotropy}{%
\paragraph{5.7 anisotropy}\label{anisotropy}}

\textbf{\texttt{Aniso.ping}} * 空間相関の強度と歪みを表す

\begin{center}\rule{0.5\linewidth}{\linethickness}\end{center}

\hypertarget{part-2-ggvast}{%
\subsection{Part 2: ggvast}\label{part-2-ggvast}}

ggvastとは,VASTの推定結果を作図するためのパッケージ.VASTではFishStatsUtilsを用いて作図をしているが,
* 後日,Save.RDataを使って作図をすることができない *
VASTやFishStatsUtilsが変更されると,これまでのコードで作図ができなくなることがある
* 軸の名前が変更できない * 推定指標値の年トレンドでは,y軸名が必ずmetric
tonnesになる * 推定密度のマップでは,NorthtingやEastingで表示される *
推定密度のマップとリジェンドが別々のファイルになる *
COGの変化がkmで表示される などの不便な点がある.ggvast
はこれらの問題を解決し,様々なハビタット,生物,研究分野でVASTを使いやすくすることを目標としている.
\#\#\# 0. ggvastのインストール

\begin{verbatim}
require(devtools)
devtools::intrall_packeage("ggvast")
\end{verbatim}

\hypertarget{vastux306eux63a8ux5b9aux7d50ux679c}{%
\subsubsection{1.
VASTの推定結果}\label{vastux306eux63a8ux5b9aux7d50ux679c}}

\hypertarget{cppux30d5ux30a1ux30a4ux30ebux306eux30d0ux30fcux30b8ux30e7ux30f3ux3092ux6307ux5b9a-1}{%
\paragraph{1.1
cppファイルのバージョンを指定}\label{cppux30d5ux30a1ux30a4ux30ebux306eux30d0ux30fcux30b8ux30e7ux30f3ux3092ux6307ux5b9a-1}}

\end{document}
